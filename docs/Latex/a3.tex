% !TEX encoding = UTF-8
%Koma article
\documentclass[fontsize=12pt,paper=letter,twoside]{scrartcl}
\usepackage{float}
\usepackage{listings}
\usepackage{makecell}

%Standard Pre-amble
\input{sty/defns.tex}
%Useful definitions
%\newcommand{\mv}[1]{\textit{m\_#1}}
%\newcommand{\cv}[1]{\textit{c\_#1}}
%\newcommand{\degree}[1]{^{\circ}\mathrm{#1}}
%\newcommand{\comment}[1]{{\footnotesize \quad\texttt{--}\textrm{#1}}}


%For Code Stylings
\usepackage{listings}
\usepackage{color}
\usepackage{hyperref}

\definecolor{dkgreen}{rgb}{0,0.6,0}
\definecolor{gray}{rgb}{0.5,0.5,0.5}
\definecolor{mauve}{rgb}{0.58,0,0.82}

\lstset{frame=tb,
  language=Java,
  aboveskip=3mm,
  belowskip=3mm,
  showstringspaces=false,
  columns=flexible,
  basicstyle={\small\ttfamily},
  numbers=none,
  numberstyle=\tiny\color{gray},
  keywordstyle=\color{blue},
  commentstyle=\color{dkgreen},
  stringstyle=\color{mauve},
  breaklines=true,
  breakatwhitespace=true,
  tabsize=3
}

% Set the header
\ihead[]{\small EECS4313 Assignment-3}


%%%%%%%%%%%%Enter your names here%%%%%%%%
\author{Student Name | Student Number | EECS Account
\and \textbf{Edward Vaisman | 212849857 | eddyv}
\and \textbf{Robin Bandzar | 212200531 | cse23028}
\and \textbf{Kirusanth Thiruchelvam | 212918298 | kirusant}
\and \textbf{Sadman Sakib Hasan | 212497509 | cse23152}
}
%%%%%%%%%%%%%%%%%%%%%%%%%%%%%%%%

\date{\today} % Display a given date or no date

\begin{document}
\title{EECS 4313 Assignment 3 \\Data Flow Testing, Slice-Based Testing and Mutation Testing}
\maketitle

\newpage

%%%%%%%%%%%%%%%%%%%%%%%%%%%%%%%
\tableofcontents
\listoftables
\listoffigures

\newpage


\section{BORG Calendar}

\subsection{Data Flow Testing}

\subsubsection{Chosen Method}

\begin{itemize}
\item \textbf{Class}: \emph{net.sf.borg.common.DateUtil.java}
\item \textbf{Method}: \emph{minuteString(int mins)}
\item \textbf{Method Description}:
This method generate a human reable string for a particular number of minutes. It returns the string in terms of hours or minutes or both hours and mintues.
\begin{itemize}
\item \textbf{mins} - The first argument is of type integer.
\end{itemize}
\end{itemize}

\noindent Following is the code snippet of the \emph{minuteString} method:
\begin{code}
	public static String minuteString(int mins) {
		
		int hours = mins / 60;
		int minsPast = mins % 60;
		
		String minutesString;
		String hoursString;
		
		if (hours > 1) {
			hoursString = hours + " " + Resource.getResourceString("Hours");
		} else if (hours > 0) {
			hoursString = hours + " " + Resource.getResourceString("Hour");
		} else {
			hoursString = "";
		}

		if (minsPast > 1) {
			minutesString = minsPast + " " + Resource.getResourceString("Minutes");
		} else if (minsPast > 0) {
			minutesString = minsPast + " " + Resource.getResourceString("Minute");
		} else if (hours >= 1) {
			minutesString = "";
		} else {
			minutesString = minsPast + " " + Resource.getResourceString("Minutes");
		}

		// space between hours and minutes
		if (!hoursString.equals("") && !minutesString.equals(""))
			minutesString = " " + minutesString;

		return hoursString + minutesString;
	}
\end{code}

\subsubsection{Variable set}

Following are the list of variables used in the \emph{minuteString} method along with their types:

\begin{table}[h]
\centering
\begin{tabular}{|c | c | c |}
	\cline{1-3}
	\textbf{Variable Name} & \textbf{Data Type} & \textbf{Is Primitive?}\\ \hline
	mins & int & Yes \\ \hline
	hours & int & Yes \\ \hline
	minsPast & int & Yes \\ \hline
	minutesString & String & No \\ \hline
	hoursString & String & No \\ \hline
\end{tabular}
\caption {List of variables for the minuteString method}
\label{tbl:list_variables}
\end{table}

\noindent \textbf{Note}: The last two variables, \emph{minutesString} and \emph{hoursString}, are \textbf{NOT} primitive types, and will \textbf{NOT} be used to perform the data flow analysis as specified by the instructor.

\subsubsection{Program Segmentation}

The following table \ref{tbl:program_segmentation} contains the \emph{minuteString} method broken down into segments:

\begin{table}[h]
\centering
\begin{tabular}{|c | c |}
	\cline{1-2}
	public static string minuteString(int mins) \{ & A \\ \hline
	int hours = mins / 60; & \\ 
	int minsPast = mins \% 60; & B \\ \hline
	String minutesString; & \\ 
	String hoursString; & C \\ \hline
	if (hours $>$ 1) \{ & D \\ \hline
	hoursString = hours + " " + Resource.getResourceString("Hours"); & E \\ \hline
	\} else if (hours $>$ 0) \{ & F \\ \hline
	hoursString = hours + " " + Resource.getResourceString("Hour"); & G \\ \hline
	\} else \{ hoursString = ""; \} & H \\ \hline
	if (minsPast $>$ 1) \{ & I \\ \hline
	minutesString = minsPast + " " + Resource.getResourceString("Minutes"); & J \\ \hline
	\} else if (minsPast $>$ 0) \{ & K \\ \hline
	minutesString = minsPast + " " + Resource.getResourceString("Minute"); & L \\ \hline
	\} else if (hours $>=$ 1) \{ & M \\ \hline
	minutesString = ""; & N \\ \hline
	\} else \{ & \\ 
	minutesString = minsPast + " " + Resource.getResourceString("Minutes"); & O \\
	\} & \\ \hline
	if (!hoursString.equals("") \&\& !minutesString.equals("")) & P \\ \hline
	minutesString = " " + minutesString; & Q \\ \hline
	return hoursString + minutesString; & R \\ \hline
\end{tabular}
\caption {Program Segmentation for the minuteString method}
\label{tbl:program_segmentation}
\end{table}

\newpage
\subsubsection{Program graph}
The following diagram \ref{fig:cfg} represents the control flow graph for the \emph{minuteString} method:

\begin{figure}[!htb]
\begin{center}
\includegraphics[width=.99\textwidth]{images/dft/cfg_drawio.pdf}
\end{center}
\caption{Control Flow Graph for the minuteString method}
\label{fig:cfg}
\end{figure}

\clearpage
\newpage
\subsubsection{Data Flow Analysis}

\begin{enumerate}
\item \textbf{du-paths for \emph{mins}}
\begin{itemize}
\item \textbf{All-Defs}: \emph{AB}
\item \textbf{All-C-Uses/Some-P-Uses}: \emph{AB}
\item \textbf{All-P-Uses/Some-C-Uses}: \emph{AB}
\item \textbf{All-Uses}: \emph{AB}
\end{itemize}

\item \textbf{du-paths for \emph{hours}}
\begin{itemize}
\item \textbf{All-Defs}: \emph{BCD}
\item \textbf{All-C-Uses/Some-P-Uses}: \emph{BCDE, BCDFG}
\item \textbf{All-P-Uses/Some-C-Uses}: \emph{BCD, BCDF, BCDEIKM, BCDFGIKM, BCDFHIKM}
\item \textbf{All-Uses}: \emph{BCD, BCDE, BCDF, BCDEIKM, BCDFG, BCDFGIKM, BCDFHIKM}
\end{itemize}

\item \textbf{du-paths for \emph{minsPast}}
\begin{itemize}
\item \textbf{All-Defs}: \emph{BCDEI}
\item \textbf{All-C-Uses/Some-P-Uses}: \emph{BCDEIJ, BCDFGIJ, BCDFHIJ, BCDEIKL, BCDFGIKL, BCDFHIKL, BCDEIKMO, BCDFGIKMO, BCDFHIKMO}
\item \textbf{All-P-Uses/Some-C-Uses}: \emph{BCDEI, BCDFGI, BCDFHI, BCDEIK, BCDFGIK, BCDFHIK}
\item \textbf{All-Uses}: \emph{BCDEI, BCDEIJ, BCDFGIJ, BCDFHIJ, BCDEIKL, BCDFGIKL, BCDFHIKL, BCDEIKMO, BCDFGIKMO, BCDFHIKMO, BCDFGI, BCDFHI, BCDEIK, BCDFGIK, BCDFHIK}
\end{itemize}

\end{enumerate}


\subsubsection{Coverage Metrics}

Following are the existing test cases from Assignment-2 and their coverage metrics using the data flow analysis for the \emph{minuteString} method:

\begin{enumerate}

\item \textbf{Test Case 1}:
\begin{itemize}
\item \textbf{Input}: mins = 60
\item \textbf{Path}: \emph{ABCDEGIKMNPR}
\item \textbf{Coverage}: 100*(12/18) = 66.67\%
\end{itemize}

\item \textbf{Test Case 2}:
\begin{itemize}
\item \textbf{Input}: mins = 61
\item \textbf{Path}: \emph{ABCDFGIKLPQR}
\item \textbf{Coverage}: 100*(14/18) = 77.78\%
\end{itemize}

\item \textbf{Test Case 3}:
\begin{itemize}
\item \textbf{Input}: mins = 75
\item \textbf{Path}: \emph{ABCDFGIJPQR}
\item \textbf{Coverage}: 100*(15/18) = 83.33\%
\end{itemize}

\item \textbf{Test Case 4}:
\begin{itemize}
\item \textbf{Input}: mins = 180
\item \textbf{Path}: \emph{ABCDEIKMNPR}
\item \textbf{Coverage}: 100*(16/18) = 88.88\%
\end{itemize}

\item \textbf{Test Case 5}:
\begin{itemize}
\item \textbf{Input}: mins = 121
\item \textbf{Path}: \emph{ABCDEIKLPQR}
\item \textbf{Coverage}: 100*(16/18) = 88.88\%
\end{itemize}

\item \textbf{Test Case 6}:
\begin{itemize}
\item \textbf{Input}: mins = 145
\item \textbf{Path}: \emph{ABCDEIJPQR}
\item \textbf{Coverage}: 100*(16/18) = 88.88\%
\end{itemize}

\item \textbf{Test Case 7}:
\begin{itemize}
\item \textbf{Input}: mins = 0
\item \textbf{Path}: \emph{ABCDFHIOPR}
\item \textbf{Coverage}: 100*(18/18) = 100.0\%
\end{itemize}

\end{enumerate}

\noindent \textbf{Rationale}: The testing technique used for testing this method is \emph{Equivalence Class Testing}. It is a suitable technique for this method since the argument is an integer which is an independent variable and the input range can be partitioned assuring disjointness and non-redundancy between each partition set. We have chosen the partition integer range based on usages of minute, minutes, hour, and hours. In order to partition the integer argument into hours and minutes, we divide the minutes by 60 to get the range of hours and the remainder (minutes \% 60) to get the range of the minutes. By covering all the cases, where the given input returns a concatenation of hours and minutes string, we were able to reach a 100\% coverage for this method.

\newpage
\subsection{Slice Testing}

\subsubsection{Chosen Method for Testing}

\begin{itemize}
\item \textbf{Class}: \emph{net.sf.borg.common.DateUtil.java}
\item \textbf{Method}: \emph{minuteString(int mins)}
\item \textbf{Method Description}:
This method generates a human readable string for a particular number of minutes. It returns the string in terms of hours or minutes or both hours and minutes.
\begin{itemize}
\item \textbf{mins} - The first argument is of type integer.
\end{itemize}
\end{itemize}

\noindent Following is the code of the \emph{minuteString} method:
\begin{lstlisting}[numbers=left,firstnumber=100]
  public static String minuteString(int mins) {
    
    int hours = mins / 60;
    int minsPast = mins % 60;
    
    String minutesString;
    String hoursString;
    
    if (hours > 1) {
      hoursString = hours + " " + Resource.getResourceString("Hours");
    } else if (hours > 0) {
      hoursString = hours + " " + Resource.getResourceString("Hour");
    } else {
      hoursString = "";
    }

    if (minsPast > 1) {
      minutesString = minsPast + " " + Resource.getResourceString("Minutes");
    } else if (minsPast > 0) {
      minutesString = minsPast + " " + Resource.getResourceString("Minute");
    } else if (hours >= 1) {
      minutesString = "";
    } else {
      minutesString = minsPast + " " + Resource.getResourceString("Minutes");
    }

    // space between hours and minutes
    if (!hoursString.equals("") && !minutesString.equals(""))
      minutesString = " " + minutesString;

    return hoursString + minutesString;
  }
\end{lstlisting}

\subsubsection{Backward Slicing}
Backward slicing is in the form of S(v,n) where the slices are code fragments that contribute to variable v at statement n. Slices are only done for primitive values and their All-defs and P-use paths defined in the data flow analysis part.\\

S(hours, 102)
\begin{lstlisting}[numbers=left,firstnumber=100]
  public static String minuteString(int mins) {
    
    int hours = mins / 60;
\end{lstlisting}  


S(minsPast, 103)
\begin{lstlisting}[numbers=left,firstnumber=100]
  public static String minuteString(int mins) {
    
    
    int minsPast = mins % 60;
\end{lstlisting} 
The following test case covers the two slices listed above and covers the All-def, P-use for \emph{mins}.
\begin{code}
  assertEquals("1 Hour",DateUtil.minuteString(60));
\end{code}

\break
S(hours, 108)
\begin{lstlisting}[numbers=left,firstnumber=100]
  public static String minuteString(int mins) {
    
    int hours = mins / 60;
    
    


    
    if (hours > 1) {
    }
\end{lstlisting}


S(hours, 120)
\begin{lstlisting}[numbers=left,firstnumber=100]
  public static String minuteString(int mins) {
    
    int hours = mins / 60;
    int minsPast = mins % 60;
    
    
    
    
    
     
    
     
    
     
    

    if (minsPast > 1) {
      
    } else if (minsPast > 0) {
      
    } else if (hours >= 1) {
    }
\end{lstlisting}
The following test case covers the previous two slices for \emph{hours}.
\begin{code}
assertEquals("3 Hours",DateUtil.minuteString(180));
\end{code}
S(minsPast, 116)
\begin{lstlisting}[numbers=left,firstnumber=100]
  public static String minuteString(int mins) {
    
    
    int minsPast = mins % 60;
    
    
    
    
    
     
    
     
    
     
    

    if (minsPast > 1) {
      
    }
\end{lstlisting}
The following test case covers the previous slice for \emph{minsPast}.
\begin{code}
assertEquals("1 Hour 15 Minutes",DateUtil.minuteString(75));
\end{code}
S(hours, 110)
\begin{lstlisting}[numbers=left,firstnumber=100]
  public static String minuteString(int mins) {
    
    int hours = mins / 60;
    
    
    
    
    
    if (hours > 1) {
      
    } else if (hours > 0) {
    
    }
\end{lstlisting}
The following test case covers the previous slice for \emph{hours}.
\begin{code}
assertEquals("1 Hour 1 Minute",DateUtil.minuteString(61));
\end{code}

S(minsPast, 118)
\begin{lstlisting}[numbers=left,firstnumber=100]
  public static String minuteString(int mins) {
    
    
    int minsPast = mins % 60;
    
    
    
    
    
     
    
     
    
     
    

    if (minsPast > 1) {
      
    } else if (minsPast > 0) {
      
    
    }
\end{lstlisting}
The following test case covers the previous slice for \emph{minsPast}.
\begin{code}
assertEquals("1 Hour 1 Minute",DateUtil.minuteString(61));
\end{code}

S(hours, 112)
\begin{lstlisting}[numbers=left,firstnumber=100]
  public static String minuteString(int mins) {
    
    int hours = mins / 60;
    
    
    
    
    
    if (hours > 1) {
      
    } else if (hours > 0) {
    
    } else {
    
    }
\end{lstlisting}
The following test case covers the previous slice for \emph{hours}.
\begin{code}
assertEquals("0 Minutes",DateUtil.minuteString(0)); 
\end{code}

S(hours, 122)
\begin{lstlisting}[numbers=left,firstnumber=100]
  public static String minuteString(int mins) {
    
    int hours = mins / 60;
    int minsPast = mins % 60;
    
    
    
    
    
     
    
     
    
     
    

    if (minsPast > 1) {
      
    } else if (minsPast > 0) {
      
    } else if (hours >= 1) {
    
    } else {
    
    }
\end{lstlisting}
S(minsPast, 122)
\begin{lstlisting}[numbers=left,firstnumber=100]
  public static String minuteString(int mins) {
    
    int hours = mins / 60;
    int minsPast = mins % 60;
    
    
    
    
    
     
    
     
    
     
    

    if (minsPast > 1) {
      
    } else if (minsPast > 0) {
      
    } else if (hours >= 1) {
    
    } else {
    
    }
\end{lstlisting}
The following test cases covers the previous two slices.
\begin{code}
assertEquals("0 Minutes",DateUtil.minuteString(0));
\end{code}

This concludes all the backward slices related to the All-defs and P-uses of the primitive types in the \emph{minuteString} function.

\newpage
\subsection{Mutation Testing}
Mutation tests were run using the previous unit test suite that we created for assignment 2. The program used to run the Mutation tests was \emph{Eclipse} with the \emph{Pitclipse} plugin. The three methods that we tested are listed with their results in the following subsections. 

\subsubsection{minuteString Function}


\begin{figure}[!htb]
\begin{center}
\includegraphics[width=.99\textwidth]{images/MutationTesting/minuteStringCode.png}
\end{center}
\caption{Code for the minuteString method}
\label{fig:minuteStringCode}
\end{figure}

\clearpage
\begin{figure}[!htb]
\begin{center}
\includegraphics[width=.99\textwidth]{images/MutationTesting/minuteStringMutant.png}
\end{center}
\caption{Mutations for the minuteString method}
\label{fig:minuteStringMutant}
\end{figure}

As one can see in Figures \ref{fig:minuteStringCode} and \ref{fig:minuteStringMutant} that the previous tests effectively kill all the mutants so no further changes are needed.


\subsubsection{isAfter Function}
\begin{figure}[!htb]
\begin{center}
\includegraphics[width=.99\textwidth]{images/MutationTesting/isAfterCode.png}
\end{center}
\caption{Code for the isAfter method}
\label{fig:isAfterCode}
\end{figure}

\clearpage
\begin{figure}[!htb]
\begin{center}
\includegraphics[width=.99\textwidth]{images/MutationTesting/isAfterMutant.png}
\end{center}
\caption{Mutations for the isAfter method}
\label{fig:isAfterMutant}
\end{figure}


\noindent Looking at the result of the mutation test result from Figure \ref{fig:isAfterMutant}, it is inevitable that the method \emph{isAfter} will still pass tests even after removing the lines 44-46 and 49-51. This means those lines (line 44-46 and line 49-51) can never have any affect on the expected output of the method.

\bigskip
\noindent This can be proved by adding a simple test case which creates two date objects and set their minute, hour and second to some value. No matter what value for minute, hour and second is passed with the Date object, the function will always create a Gregorian Calendar instance with values of 0 for the first date object. Similarly, it will create another Gregorian Calendar instance with values 0 for hours and seconds and 10 for minutes for the second date object.

\bigskip

\noindent \textbf{Example test case:} \label{isAfter:example_test_case}

\begin{code}
    @Test
    public void testIsAfterBug() {
    	// Both happens at the same day, but at different hours
    	// but the implementation ignores the hours, minutes
    	// and seconds
    	Date d1 = new Date( 2018 , 8 , 24 , 8 , 30 , 24 );
    	Date d2 = new Date( 2018 , 8 , 24 , 9 , 30 , 24 );
    	
    	// This works correctly. The assert returns true
    	assertFalse(DateUtil.isAfter(d1, d2));
    	// This does not work correctly. The assert returns false,
    	// when it should be true
    	assertTrue(DateUtil.isAfter(d2, d1));
    }
\end{code}

\bigskip

\noindent \textbf{Bug Report:}

\begin{itemize}
\item \textbf{Bug Report Title}: The \emph{isAfter} function ignores hours, minutes and seconds fields from the passed in date object.
\item \textbf{Reported by}: Sadman Sakib Hasan
\item \textbf{Date reported}: Saturday, March 24, 2018
\item \textbf{Program (or component) name}: BORG Calendar version 1.8.3
\item \textbf{Configuration(s)}:\\
\underline{System Info}
\begin{itemize}
\item{Operating System: Windows 10 Pro 64-bit (10.0, Build 16299)}
\item {Language: English (Regional Setting: English)}
\item {System Manufacturer: Hewlett-Packard}
\item {System Model: HP 15 TouchSmart Notebook PC}
\item {BIOS: F.10}
\item {Processor: AMD A6-5200 APU with Radeon(TM) HD Graphics (4 CPUs), ~2.0GHz }
\item {Memory: 6144MB RAM}
\item {Java Version: 1.8.0\_151-b12}
\end{itemize}
\item \textbf{Report type}: Coding Error
\item \textbf{Reproducibility}: 100\%
\item \textbf{Severity}: Medium
\item \textbf{Problem summary}: The \emph{isAfter} method automatically sets the HOUR\_OF\_DAY, MINUTE, and SECOND values of the newly created Gregorian Calendar object from the passed in dates d1 and d2 to a constant value.
\item \textbf{Problem description}: The \emph{isAfter} method in \emph{net.sf.borg.common.DateUtil} does not take into account the hours, minutes, and seconds of the given date objects when comparing the two dates. A newly Gregorian Calendar object is created for both argument, d1 and d2, and the hours, minutes and seconds are set to 0 for \emph{d1}'s calendar instance whereas the hours and seconds are set to 0 and minute set to 10 for \emph{d2}'s calendar instance. This causes an issue when there are comparison between two dates that are in the same day, and hours, minutes and seconds differ as shown in the example test case \ref{isAfter:example_test_case}.
\item \textbf{Steps to Reproduce}:
\begin{itemize}
\item Create two date objects, d1 and d2, with the day set to the same value and hours, minutes and seconds set to different values.
\item Compare the two dates using \emph{isAfter} function.
\item The \emph{isAfter} function will always return false regardless of the fact if d1 is ahead on hours/minutes/seconds or vice versa.
\end{itemize}
\item \textbf{New or old bug}: New
\end{itemize}


\subsubsection{sendMsg Function}
\begin{figure}[!htb]
\begin{center}
\includegraphics[width=.99\textwidth]{images/MutationTesting/sendMsgCode.png}
\end{center}
\caption{Code for the sendMsg method}
\label{fig:sendMsgCode}
\end{figure}

\begin{figure}[!htb]
\begin{center}
\includegraphics[width=.99\textwidth]{images/MutationTesting/sendMsgMutant.png}
\end{center}
\caption{Mutations for the sendMsg method}
\label{fig:sendMsgMutant}
\end{figure}

The results show that not all mutants have been killed. From Figure \ref{fig:sendMsgCode} we can see that our mutation testing results can be possibly improved if more tests on the server and socket state are created. 

\clearpage
\bigskip
\fontsize{14}{5}\textbf{After Additional Test Cases}

\begin{figure}[!htb]
\begin{center}
\includegraphics[width=.99\textwidth]{images/MutationTesting/sendMsgCodeAfter.png}
\end{center}
\caption{Updated code results for the sendMsg method}
\label{fig:sendMsgCodeAfter}
\end{figure}

\begin{figure}[!htb]
\begin{center}
\includegraphics[width=.99\textwidth]{images/MutationTesting/sendMsgMutantAfter.png}
\end{center}
\caption{Updated mutations for the sendMsg method}
\label{fig:sendMsgMutantAfter}
\end{figure}

\clearpage
The test case shown below was added and improved results to what is shown in Figures \ref{fig:sendMsgCodeAfter} and \ref{fig:sendMsgMutantAfter}. This test case kills the mutant at lines 44, 48 without increasing line coverage. Thus, showing how test cases built for coverage alone are not sufficient. \\

The remaining bugs all relate to the \emph{close} function of the Socket object instantiated in the \emph{sendMsg} method. When Pitclipse mutates the \emph{close} function and removes it, this action causes a timeout. Which is why the \emph{sendMsg} method is defensively programmed to ensure that the socket connection is closed after the message is sent. Due to the amount of defensive programming and case of timeouts, it is not possible to achieve 100\% only using mutation testing for the method sendMsg.

\begin{lstlisting}[caption={Additional test case for sendMsg},label={list:sendMsg}]
    @Test
    public void checkServerAlive() {
    String msg = null;
    SocketServer ss;
    String response;
        try {
            ss = new SocketServer(2922, this);
            response = SocketClient.sendMsg("localhost", 2922, msg);
            assertTrue(ss.isAlive());
        } catch (IOException e) {
            // TODO Auto-generated catch block
            e.printStackTrace();
        }
    }
\end{lstlisting}

\newpage
\section{JPetStore}

After exploring the JPetStore system, we came up with some realistic non-trivial test scenarios that can be carried out for load testing using JMeter. The following subsections cover each scenarios, description on how it was load tested and the result analysis of the load test.

\bigskip
\noindent Following are the system specifications for which the load test was conducted under:
\begin{itemize}
\item \textbf{Operating System}: Windows 10 Pro 64-bit (10.0, Build 16299)
\item \textbf{Language}: English (Regional Setting: English)
\item \textbf{System Manufacturer}: Hewlett-Packard
\item \textbf{System Model}: HP 15 TouchSmart Notebook PC
\item \textbf{BIOS}: F.10
\item \textbf{Processor}: AMD A6-5200 APU with Radeon(TM) HD Graphics (4 CPUs), ~2.0GHz
\item \textbf{Memory}: 6144MB RAM
\item \textbf{Java Version}: 1.8.0\_151-b12
\item \textbf{Apache Tomcat Version}: 7.0.85
\item \textbf{JMeter Version}: 2.11 r1554548
\end{itemize}

\newpage
\subsection{Scenario 1: Returning User}

\subsubsection{Overview}
The first test scenario is for an existing user in the system. The testing scenario will consist of a returning user logging in, selecting one of each of the 5 possible items sold in JPetStore, adding the items to the cart, performing a checkout of the cart and finally logging out. The following describes an exact breakdown of the steps the load test will carry out:

\begin{itemize}
\item Access the JPetStore Homepage (\url{http://localhost:8080/jpetstore/}).
\item Click \emph{Enter the Store}.
\item Click on the \emph{Sign in} button.
\item Enter sign-in credentials and click \emph{Login} By default, we will be using \emph{j2ee} user for this scenario.
\item Go to the \emph{Fish} section, select a fish item, add it to the cart and return to the main menu.
\item Go to the \emph{Dogs} section, select a dog item, add it to the cart and return to the main menu.
\item Go to the \emph{Reptiles} section, select a reptile item, add it to the cart and return to the main menu.
\item Go to the \emph{Cats} section, select a cat item, add it to the cart and return to the main menu.
\item Go to the \emph{Birds} section, select a bird item, add it to the cart.
\item Proceed to checkout and follow the steps until the order has been placed.
\item Return back to the main menu and sign out.
\end{itemize}

\newpage
\bigskip
\noindent The following images depicts the Recording Controller for the test case scenario and the pages our load test will navigate through per each iteration:

\begin{figure}[!htb]
\begin{center}
\includegraphics[width=.6\textwidth]{../../load-test/test-plans/returning-user/rc.png}
\end{center}
\caption{Recording Controller for Returning User Scenario}
\label{fig:ruser:rc}
\end{figure}

\clearpage
\subsubsection{Load Test Properties}
Following are the load test properties applied for testing this scenario:
\begin{itemize}
\item \textbf{Number of Thread (users)}: 5
\item \textbf{Ramp-up Period (in seconds)}: 10
\item \textbf{Loop Count}: 30
\item \textbf{Thread Delay (in milliseconds)}: 1000
\end{itemize}

\subsubsection{Executing Load Test}
After setting up the load test plan using JMeter, we executed the test. The test run was for approximately 18 minutes completing all 30 iterations. The following diagrams show the result tree of the test run.

\begin{figure}[!htb]
\begin{center}
\includegraphics[width=.9\textwidth]{../../load-test/test-plans/returning-user/result-tree.png}
\end{center}
\caption{View Results Tree for Returning User Scenario}
\label{fig:ruser:view_result_tree}
\end{figure}

\subsubsection{Analysing Load Test}
The following statistics were gathered from the Apache access logs. The access log file was parsed to get the start of the load test and the end of the load test and the total time of the load test execution.

\bigskip
\noindent Within that timeframe, the total number of requests was found by simply executing: \emph{wc -l ruser\_log.txt}. The total number of GET and POST requests were found by executing \emph{grep -c -w "GET" ruser\_log.txt} and \emph{grep -c -w "POST" ruser\_log.txt} respectively.

\begin{itemize}
\item \textbf{Test Duration}: Approximately 18 minutes
\item \textbf{Total Number of Requests}: 4714
\item \textbf{Request Rate}: 4.36 requests/second
\item \textbf{Number of GET Requests}: 4399 (93\% of all requests)
\item \textbf{Number of POST Requests}: 315 (7\% of all requests)
\item \textbf{Success HTTP Codes}: 200 (Success) and 302 (Found)
\item \textbf{Failure HTTP Codes}: None
\end{itemize}

\bigskip
\noindent Following is a snapshot of the Apache access log:
\begin{figure}[!htb]
\begin{center}
\includegraphics[width=.6\textwidth]{../../load-test/test-plans/returning-user/access-log.png}
\end{center}
\caption{Apache Access Log Snapshot for Returning User Scenario}
\label{fig:ruser:access_log}
\end{figure}

\clearpage
\bigskip
\noindent Following is a snapshot of the Windows Performance Monitor and Java Monitor Console during the load test execution:

\begin{figure}[!htb]
\begin{center}
\includegraphics[width=.7\textwidth]{../../load-test/test-plans/returning-user/perfmon-ru.png}
\end{center}
\caption{Windows Performance Monitor Snapshot for Returning User Scenario}
\label{fig:ruser:perfmon}
\end{figure}

\begin{figure}[!htb]
\begin{center}
\includegraphics[width=.7\textwidth]{../../load-test/test-plans/returning-user/jconsole-ru.png}
\end{center}
\caption{JConsole Snapshot for Returning User Scenario}
\label{fig:ruser:jconsole}
\end{figure}

\smallskip
\noindent \textbf{Conclusion}: The load test scenario for returning user made about 4714 requests and the test ran for approximately 18 minutes. Despite some natural high spikes on the performance monitor and java monitor, the load test was carried out successfully without any crashes or unexpected behaviour in the application.


\newpage
\subsection{Scenario 2: New User}

\subsubsection{Overview}
The second test scenario is for a new user in the system. The testing scenario will consist of registering a new user, selecting one of each of the 5 possible items sold in JPetStore, adding the items to the cart, performing a checkout of the cart and finally logging out. The following describes an exact breakdown of the steps the load test will carry out:

\begin{itemize}
\item Access the JPetStore Homepage (\url{http://localhost:8080/jpetstore/}).
\item Click \emph{Enter the Store}.
\item Click on the \emph{Sign in} button.
\item Click \emph{Register now}.
\item Enter the sign-up credentials and click \emph{Create Account}. The username and password for signing up would be loaded from a CSV file, whereas the other fields will be supplied the value of \emph{abc}.
\item Go to the \emph{Fish} section, select a fish item, add it to the cart and return to the main menu.
\item Go to the \emph{Dogs} section, select a dog item, add it to the cart and return to the main menu.
\item Go to the \emph{Reptiles} section, select a reptile item, add it to the cart and return to the main menu.
\item Go to the \emph{Cats} section, select a cat item, add it to the cart and return to the main menu.
\item Go to the \emph{Birds} section, select a bird item, add it to the cart.
\item Proceed to checkout and follow the steps until the order has been placed.
\item Return back to the main menu and sign out.
\end{itemize}

\newpage
\bigskip
\noindent The following images depicts the Recording Controller for the test case scenario and the pages our load test will navigate through per each iteration:

\begin{figure}[!htb]
\begin{center}
\includegraphics[width=.5\textwidth]{../../load-test/test-plans/new-user/rc.png}
\end{center}
\caption{Recording Controller for New User Scenario}
\label{fig:ruser:rc}
\end{figure}

\bigskip
\noindent The following images depicts the snapshot of the CSV file containing usernames and password for registering new users. On the POST request for registering a new user we load the credentials and use them through variable names such as \$\{username\} and \$\{password\}.

\begin{figure}[!htb]
\begin{center}
\includegraphics[width=.6\textwidth]{../../load-test/test-plans/new-user/csv.png}
\end{center}
\caption{CSV Credential File for New User Scenario}
\label{fig:ruser:csv}
\end{figure}

\begin{figure}[!htb]
\begin{center}
\includegraphics[width=.7\textwidth]{../../load-test/test-plans/new-user/post-request.png}
\end{center}
\caption{HTTP Post Request for New User Scenario}
\label{fig:ruser:csv}
\end{figure}

\clearpage
\subsubsection{Load Test Properties}
Following are the load test properties applied for testing this scenario:
\begin{itemize}
\item \textbf{Number of Thread (users)}: 5
\item \textbf{Ramp-up Period (in seconds)}: 5
\item \textbf{Loop Count}: 30
\item \textbf{Thread Delay (in milliseconds)}: 1000
\end{itemize}

\subsubsection{Executing Load Test}
After setting up the load test plan using JMeter, we executed the test. The test run was for approximately 15 minutes completing all 30 iterations. The following diagrams show the result tree of the test run.

\begin{figure}[!htb]
\begin{center}
\includegraphics[width=.9\textwidth]{../../load-test/test-plans/new-user/result-tree.png}
\end{center}
\caption{View Results Tree for New User Scenario}
\label{fig:nuser:view_result_tree}
\end{figure}

\subsubsection{Analysing Load Test}
The following statistics were gathered from the Apache access logs. The access log file was parsed to get the start of the load test and the end of the load test and the total time of the load test execution.

\bigskip
\noindent Within that timeframe, the total number of requests was found by simply executing: \emph{wc -l nuser\_log.txt}. The total number of GET and POST requests were found by executing \emph{grep -c -w "GET" nuser\_log.txt} and \emph{grep -c -w "POST" nuser\_log.txt} respectively.

\begin{itemize}
\item \textbf{Test Duration}: Approximately 15 minutes
\item \textbf{Total Number of Requests}: 7500
\item \textbf{Request Rate}: 8.33 requests/second
\item \textbf{Number of GET Requests}: 7000 (93\% of all requests)
\item \textbf{Number of POST Requests}: 500 (7\% of all requests)
\item \textbf{Success HTTP Codes}: 200 (Success) and 302 (Found)
\item \textbf{Failure HTTP Codes}: None
\end{itemize}

\bigskip
\noindent Following is a snapshot of the Apache access log:
\begin{figure}[!htb]
\begin{center}
\includegraphics[width=.8\textwidth]{../../load-test/test-plans/new-user/access-log.png}
\end{center}
\caption{Apache Access Log Snapshot for New User Scenario}
\label{fig:nuser:access_log}
\end{figure}

\clearpage
\bigskip
\noindent Following is a snapshot of the Windows Performance Monitor and Java Monitor Console during the load test execution:

\begin{figure}[!htb]
\begin{center}
\includegraphics[width=.7\textwidth]{../../load-test/test-plans/new-user/perfmon-nu.png}
\end{center}
\caption{Windows Performance Monitor Snapshot for New User Scenario}
\label{fig:nuser:perfmon}
\end{figure}

\begin{figure}[!htb]
\begin{center}
\includegraphics[width=.7\textwidth]{../../load-test/test-plans/new-user/jconsole-nu.png}
\end{center}
\caption{JConsole Snapshot for New User Scenario}
\label{fig:nuser:jconsole}
\end{figure}

\smallskip
\noindent \textbf{Conclusion}: The load test scenario for new user made about 7500 requests and the test ran for approximately 15 minutes. Despite some natural high spikes on the performance monitor and java monitor, the load test was carried out successfully without any crashes or unexpected behaviour in the application.

\end{document}
